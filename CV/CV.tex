%% start of file `template.tex'.
%% Copyright 2006-2015 Xavier Danaux (xdanaux@gmail.com), 2020-2022 moderncv maintainers (github.com/moderncv).
%
% This work may be distributed and/or modified under the
% conditions of the LaTeX Project Public License version 1.3c,
% available at http://www.latex-project.org/lppl/.


\documentclass[10pt,a4paper,sans]{moderncv} 
% possible options include font size ('10pt', '11pt' and '12pt'), paper size ('a4paper', 'letterpaper', 'a5paper', 'legalpaper', 'executivepaper' and 'landscape') and font family ('sans' and 'roman')

% moderncv themes
% moderncvstylebanking uses modencvheadiii.sty
\moderncvstyle{banking} % style options are 'casual' (default), 'classic', 'banking', 'oldstyle' and 'fancy'
\moderncvcolor{black} % colour options 'black', 'blue' (default), 'burgundy', 'green', 'grey', 'orange', 'purple' and 'red'
%\renewcommand{\familydefault}{\sfdefault}         % to set the default font; use '\sfdefault' for the default sans serif font, '\rmdefault' for the default roman one, or any tex font name
%\nopagenumbers{}                                  % uncomment to suppress automatic page numbering for CVs longer than one page
% Page numbers in banking style

% adjust the page margins
\usepackage[top=3cm, bottom=2cm]{geometry}
%\usepackage[scale=0.75]{geometry} % default scale=0.75
%\setlength{\footskip}{136.00005pt}                 % depending on the amount of information in the footer, you need to change this value. comment this line out and set it to the size given in the warning
\recomputelengths
%\setlength{\hintscolumnwidth}{3cm}                % if you want to change the width of the column with the dates
%\setlength{\makecvheadnamewidth}{10cm}            % for the 'classic' style, if you want to force the width allocated to your name and avoid line breaks. be careful though, the length is normally calculated to avoid any overlap with your personal info; use this at your own typographical risk...

% font loading
% for luatex and xetex, do not use inputenc and fontenc
% See https://tex.stackexchange.com/a/496643
\ifxetexorluatex
  \usepackage{luatexja-preset}
  \usepackage{fontspec}
  \usepackage{unicode-math}
  \defaultfontfeatures{Ligatures=TeX}
  \setmainfont{Latin Modern Roman}
  \setsansfont{Latin Modern Sans}
  \setmonofont{Latin Modern Mono}
  \setmathfont{Latin Modern Math} 
\else
  \usepackage[utf8]{inputenc}
  \usepackage[T1]{fontenc}
  \usepackage{lmodern}


% document language
\usepackage[english]{babel}  % FIXME: using Spanish breaks moderncv

% biblatex
% safeinputenc is needed
\usepackage[backend=biber, sorting=none, maxbibnames=99, safeinputenc, url=false]{biblatex}
\DeclareFieldFormat{pages}{#1}
\DeclareFieldFormat{postnote}{#1}
\DeclareFieldFormat{multipostnote}{#1} % citeのときにp.を表示させない。
\DeclareFieldFormat[article]{title}{\mkbibemph{#1}} % titleのフォントをイタリックにする。
% \DeclareFieldFormat*{title}{\mkbibquote{#1\adddot}}  % タイトルにダブルクオーテーションをつける
\DeclareFieldFormat{journaltitle}{\textup{#1}\addcomma\space} % journalのフォントをローマンフォントにする。
\addbibresource{publications.bib}

% load after babel and biblatex
% biblatex recommends loading csquotes with babel
\usepackage{csquotes}

% personal data
\firstname{Ryo}
\familyname{ISHIZUKA}
\renewcommand*{\namefont}{\huge\bfseries\upshape}
% \renewcommand*{\namefont}{\Size\Font\upshape}
% \Font = \mdseries \bfseries
%\name{Ryo}{Ishizuka}  % {First name}{Family name}

\title{石塚 伶}  % optional, remove/comment the line if not wanted
\renewcommand*{\titlefont}{\huge\mdseries\upshape}

% \born{3 December 1999}    % {day month year} optional, remove / comment the line if not wanted

\address{Tokyo Institute of Technology}{2-12-1 Ookayama 152-8551 Meguro-ku Tokyo}{Japan}
\renewcommand*{\addressfont}{\large\mdseries\upshape}
% \address{2-12-1 Ookayama}{152-8551 Meguro-ku Tokyo}{Japan}
% {street and number}{postcode city}{country}
% optional, remove/comment the line if not wanted; the "postcode city" and "country" arguments can be omitted or provided empty


%\phone[mobile]{+1~(234)~567~890} % optional, remove/comment the line if not wanted; the optional "type" of the phone can be "mobile" (default), "fixed" or "fax"
%\phone[fixed]{+2~(345)~678~901}
%\phone[fax]{+3~(456)~789~012}

\email{ishizuka.r.ac@m.titech.ac.jp} % optional, remove/comment the line if not wanted


% \homepage{ryo1203.github.io} % optional, remove/comment the line if not wanted

% Social icons
% \social[linkedin]{john.doe} % optional, remove/comment the line if not wanted
% \social[Twitter]{ji\_doe} % optional, remove/comment the line if not wanted
% \social[github]{jdoe} % optional, remove/comment the line if not wanted
% \social[stackoverflow]{0000000/johndoe} % optional, remove / comment the line if not wanted
% \social[orcid]{0000-0000-000-000} % optional, remove/comment the line if not wanted
% \social[researchgate]{jdoe} % optional, remove/comment the line if not wanted
% \social[researcherid]{jdoe} % optional, remove/comment the line if not wanted
% \social[googlescholar]{googlescholarid} % optional, remove/comment the line if not wanted

\extrainfo{\href{https://ryo1203.github.io}{https://ryo1203.github.io}} % optional, remove/comment the line if not wanted


% \photo[64pt][0.4pt]{ryo}  % optional, remove/comment the line if not wanted; '64pt' is the height the picture must be resized to, 0.4pt is the thickness of the frame around it (put it to 0pt for no frame) and 'picture' is the name of the picture file

% Photo Include in banking style
% graphics
% \RequirePackage[dvipdfmx]{graphicx}
% % dvipdfmxを入れないと写真が出力されない
% \patchcmd{\makehead}
%   {\hfil}
%   {\hspace*{0.15\textwidth}}
%   {}
%   {}
% \patchcmd{\makehead}
%   {\setlength{\makeheaddetailswidth}{0.8\textwidth}}
%   {\setlength{\makeheaddetailswidth}{0.67\textwidth}}
%   {}
%   {}
% \patchcmd{\makehead}
%   {\\[2.5em]}
%   {\hfil\raisebox{-.7cm}{\framebox{\includegraphics[width=\@photowidth]{\@photo}}}\\[2.5em]}
%   {}
%   {}

%\quote{Some quote} % optional, remove/comment the line if not wanted

% bibliography adjustments (only useful if you make citations in your resume, or print a list of publications using BibTeX)
%   to show numerical labels in the bibliography (default is to show no labels)
%\makeatletter\renewcommand*{\bibliographyitemlabel}{\@biblabel{\arabic{enumiv}}}\makeatother
%\renewcommand*{\bibliographyitemlabel}{[\arabic{enumiv}]}
%   to redefine the bibliography heading string ("Publications")
%\renewcommand{\refname}{Articles}

% Reverse numbering in the publications list
\newcounter{entrycount}
\AtDataInput{\stepcounter{entrycount}}
\DeclareFieldFormat{labelnumber}{\mkrevbibnum{#1}}
\newcommand{\mkrevbibnum}[1]{\number\numexpr\value{entrycount}+1-#1}

% bibliography with multiple entries
%\usepackage{multibib}
%\newcites{book,misc}{{Books},{Others}}
%----------------------------------------------------------------------------------
%            content
%----------------------------------------------------------------------------------
\begin{document}
%\begin{CJK*}{UTF8}{gbsn} % to typeset your resume in Chinese using CJK
%-----       resume       ---------------------------------------------------------
\makecvtitle

\section{Education}
\cventry{Apr 2024--Current}{Doctor of Science in Mathematics}{Tokyo Institute of Technology}{Tokyo, Japan}{}{Supervisor: Kazuma Shimomoto}
\cventry{Apr 2022--Mar 2024}{Master of Science in Mathematics}{Tokyo Institute of Technology}{Tokyo, Japan}{}{Supervisor: Kazuma Shimomoto (2nd year), Fumiharu Kato (1st year)}
\cventry{Apr 2018--Mar 2022}{Bachelor of Science in Mathematics}{Tokyo Institute of Technology}{Tokyo, Japan}{}{Supervisor: Fumiharu Kato}
%\cventry{year--year}{Degree}{Institution}{City}{\textit{Grade}}{Description} % arguments 3 to 6 can be left empty

\section{Professional Position}
\cventry{}{Tokyo Institute of Technology}{JSPS Research Fellow (DC1)}{Apr 2024--Mar 2027}{}{Host Researcher: Kazuma Shimomoto}

\section{Research Interests}
Commutative algebra in mixed characteristic (via arithmetic methods such as perfectoid rings, prismatic cohomology, and almost
mathematics).



% Publications from a BibTeX file without multibib
%  for numerical labels: \renewcommand{\bibliographyitemlabel}{\@biblabel{\arabic{enumiv}}}% CONSIDER MERGING WITH PREAMBLE PART
% \renewcommand{\refname}{Preprints} %  to redefine the heading string ("Publications"): 
% \nocite{*}
% \bibliographystyle{unsrt} % written sort
% \bibliography{publications} % 'publications' is the name of a BibTeX file



\nocite{*}
\printbibliography[title={Papers and Preprints}]


% Publications from a BibTeX file using the multibib package
%\section{Publications}
%\nocitebook{book1,book2}
%\bibliographystylebook{plain}
%\bibliographybook{publications}                   % 'publications' is the name of a BibTeX file
%\nocitemisc{misc1,misc2,misc3}
%\bibliographystylemisc{plain}
%\bibliographymisc{publications}                   % 'publications' is the name of a BibTeX file



\section{Talks}

\cvlistitem{Sep 2024. \emph{``Frobenius maps on mixed characteristic rings via prismatic cohomology''} (Poster Session), \textsf{L-functions and Motives in Niseko 2024}, Setsu Niseko and Niseko Residents Center, Japan}

\cvlistitem{Sep 2024. \emph{``A generalization of Kunz's theorem to mixed characteristic via p-adic cohomology theory''}, \textsf{MSJ Autumn Meeting 2024}, \textsf{MSJ Autumn Meeting 2024}, Osaka University, Japan}

\cvlistitem{July 2024. \emph{``Perfectoid spaces, tilts and untilts''} (Survey talk), \textsf{Atelier de Géométrie Arithmétique 2024}, RIMS (Kyoto University), Japan}

\cvlistitem{July 2024. \emph{``Prismatic approach to a mixed characteristic Kunz's theorem''}, \textsf{The 23nd Hiroshima-Sendai Workshop on Number Theory at Sendai}, Tohoku University, Japan}

\cvlistitem{June 2024. \emph{``Regularity criterion of mixed characteristic rings via prismatic cohomology''}, \textsf{Keio Algebra Seminar}, Keio University, Japan}

\cvlistitem{June 2024. \emph{``Prisms and regular local rings''}, \textsf{The 35th Seminar on Commutative Algebra in Japan}, Tokushima University, Japan}

\cvlistitem{Apr 2024. \emph{``Prisms and its application to regular rings''}, \textsf{Saturday Seminar}, Meiji University, Japan}

\cvlistitem{Mar 2024. \emph{``Perfectoid ideals and its correspondence''}, \textsf{The 20th Mathematics Conference for Young Researchers}, Hokkaido University, Japan}

\cvlistitem{Feb 2024. \emph{``Mixed characteristic Kunz's theorem with prismatic theory''}, \textsf{The 28th Conference on Algebra for Young Researchers in Japan}, Waseda University, Japan}

\cvlistitem{Dec 2023. \emph{``Commutative ring theoretic approach for the perfectoidization of semiperfectoid rings''}, \textsf{Number Theory Seminar}, Kyoto University, Japan}

\cvlistitem{Nov 2023. \emph{``Ideal correspondence between a perfectoid ring and its tilt''}, \textsf{The 44th Japan Symposium on Commutative Algebra}, LecTore Hayama, Japan}

\cvlistitem{Aug 2023. \emph{``Absolute integral closure''} (Survey talk), \textsf{The 18th Summer School on Commutative algebra}, Tokyo Institute of Technology, Japan}

\cvlistitem{Aug 2023. \emph{``On the relation between perfectoidization and \(p\)-root closure''}, \textsf{The 9th China-Japan-Korea International Conference on Ring and Module Theory}, Incheon National University, Republic of Korea}

\cvlistitem{July 2023. \emph{``On the commutative ring-theoretic structure of the perfectoidization of semiperfectoid rings''}, \textgt{The 22nd Hiroshima-Sendai Workshop on Number Theory at Hiroshima}, Hiroshima University, Japan}

\cvlistitem{July 2023. \emph{``On the application of perfectoidization to commutative algebra and its structure''}, \textgt{The 34th Seminar on Commutative Algebra in Japan}, Kitami Institute of Technology, Japan}

\cvlistitem{May 2023. \emph{``On Perfectoid(ization) and its commutative ring-theoretic properties''}, \textgt{Ookayama Youth Seminar in Algebra}, Tokyo Institute of Technology, Japan}

\cvlistitem{Mar 2023. \emph{``A mixed characteristic analogue of the perfection of rings''}, \textsf{The 11th Japan-Vietnam Joint Seminar on Commutative Algebra - by and for young mathematicians -}, Vietnam Academy of Science and Technology, Vietnam}

\cvlistitem{Mar 2023. \emph{``An explicit construction of perfectoid almost Cohen-Macaulay algebras in mixed characteristic''}, \textsf{MSJ Spring Meeting 2023}, Chuo University, Japan}

\cvlistitem{Mar 2023. \emph{``An explicit construction of perfectoid almost Cohen-Macaulay algebras''}, \textsf{The 19th Mathematics Conference for Young Researchers}, Hokkaido University, Japan}

\cvlistitem{Oct 2022. \emph{``An explicit construction of perfectoid almost Cohen-Macaulay algebras
in mixed characteristic''}, \textsf{The 43rd Japan Symposium on Commutative Algebra}, Osaka University, Japan}



\section{Membership}
\cvlistitem{Apr 2023-- . Mathematical Society in Japan (MSJ)}


\section{Languages}
\cvitemwithcomment{Japanese}{Native}{Mother tangue}
\cvitemwithcomment{English}{Intermediate}{Can read, write, and, listen but may struggle with conversation}
\cvitemwithcomment{French}{Beginner}{Can only read mathematical texts}

% \section{List item}
% \cvlistitem{Item 1}
% \cvlistitem{Item 2}
% \cvlistitem{Item 3. This item is particularly long and therefore normally spans several lines. Did you notice the indentation when the line wraps?}


% \section{Item}
% \cvitem{heading}{text}
% \cvitem{heading}{text}
% \cvitem{heading}{Text}


% \section{Item with comment}
% \cvitemwithcomment{Language 1}{Skill level}{Comment}
% \cvitemwithcomment{Language 2}{Skill level}{Comment}
% \cvitemwithcomment{Language 3}{Skill level}{Comment}
% \cvitemwithcomment{Language 4}{Skill level}{Comment}


% \section{Double item}
% \cvdoubleitem{category 1}{XXX, YYY, ZZZ}{category 4}{XXX, YYY, ZZZ}
% \cvdoubleitem{category 2}{XXX, YYY, ZZZ}{category 5}{XXX, YYY, ZZZ}
% \cvdoubleitem{category 3}{XXX, YYY, ZZZ}{category 6}{XXX, YYY, ZZZ}


% \section{List double item}
% \cvlistdoubleitem{Item 1}{Item 4}
% \cvlistdoubleitem{Item 2}{Item 5\cite{book2}}
% \cvlistdoubleitem{Item 3}{Item 6. Like item 3 in the single-column list before, this item is particularly long to wrap over several lines.}


% \section{CV column}
% \begin{cvcolumns}
%   \cvcolumn{Category 1}{\begin{itemize}\item Person 1\item Person 2\item Person 3\end{itemize}}
%   \cvcolumn{Category 2}{Amongst others:\begin{itemize}\item Person 1, and\item Person 2\end{itemize}(more upon request)}
%   \cvcolumn[0.5]{All the rest \& some more}{\textit{That} person, and \textbf{those} also (all available upon request).}
% \end{cvcolumns}


\clearpage
\end{document}

